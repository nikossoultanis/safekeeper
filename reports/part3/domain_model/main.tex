\documentclass{article}
\usepackage[LGR,T1]{fontenc}
\usepackage[utf8]{inputenc}
\usepackage[greek, english]{babel}
\usepackage{alphabeta}
\usepackage{natbib}
\usepackage{graphicx}
\usepackage{biblatex}
\addbibresource{references.bib}

\def\code#1{\texttt{#1}}

\usepackage{eso-pic}% http://ctan.org/pkg/eso-pic
\usepackage{lipsum}% http://ctan.org/pkg/lipsum

\title{Domain Model-v0.2}

\author{\\
\includegraphics[width=3in]{safeguard}\\[1ex]\\\\
}

\begin{document}

\maketitle

\newpage

Θεόδωρος Ντάκουρης - ntakouris@ceid.upatras.gr - 1054332: Editor \\
Βασίλειος Βασιλόπουλος - vvasil@ceid.upatras.gr - 1054410: Editor \\
\\

\begin{tabular}{|l|c|c|}
\hline
Όνοματεπώνυμο & email & Αριθμός μητρώου  \\
\hline
Θεόδωρος Ντάκουρης & ntakouris@ceid.upatras.gr & 1054332 \\
Βασίλειος Βασιλόπουλος & vvasil@ceid.upatras.gr &  1054410 \\
Νικόλαος Σουλτάνης & soultanis@ceid.upatras.gr & 1054319  \\
Βάιος Λασκαρέλιας & laskarelias@ceid.upatras.gr & 1054432 \\
Αντόν Παπά & papa@ceid.upatras.gr & 1054337 \\
\hline
\end{tabular}

\renewcommand{\contentsname}{Περιεχόμενα}
\tableofcontents

\section{Changelog}
\subsection{v0.2}
Καθώς έγιναν προσθήκες, η ομάδα θα προχωρήσει σε συμπλήρωση attributes και functions στο επόμενο βήμα κατόπιν τα διαγράμματα sequence.

\begin{itemize}
    \item Προσθήκη RFID Card Data, RFID Reader Data, κατόπιν συγγραφής robustnes
    \item Παρομοίως, προσθήκη DocumentAttachment, AccessLevel
    \item Αφαιρέθηκαν PIN Entry Checks, Pin Entry 
    \item Αυτές οι προσθαφαιρέσεις διαπιστώθηκαν κατά τη δημιουργία Robustness
    \item Συμπληρώθηκαν μερικές συνδέσεις composition, γιατί στην προηγούμενη έκδοση υπήρχαν μόνο aggregations στη θέση τους.
    \hline
    \item Άρχισε η προσθήκη attributes σύμφωνα με τα διαγράμματα robustness.
\end{itemize}

\section{Λεκτικές Περιγραφές}

\subsection{Document Attachment}
Κάνει encapsulate λεπτομέρειες εγγράφου που γίνεται επισύναψη.

\subsection{Access Level}
Περιλαμβάνει επίπεδο προσβασιμότητας.

\subsection{RFID Card Data}
Περιέχει μερικά δεδομένα του employee καθώς και επίπεδο πρόσβασης, ημερομηνία έκδοσης-λήξης και άλλα.

\subsection{RFID Scanner Data}
Αντιπροσωπεύει δεδομένα που αποθηκεύονται σε κάθε 'unit' που σκανάρει κάρτες. Για παράδειγμα περιέχει τον sector, access level που βρίσκεται το unit.

\subsection{RFID Card IO}
Συνοψίζει τις βασικές λειτουργίες του third party card reader που θα χρησιμοποιήσουμε. Λογικά είναι ένα απλό read και ένα write.
\subsection{Sound Player}
Ο μόνος του ρόλος είναι να παίξει τους βασικούς ήχους επιτυχίας-αποτυχίας.
\subsection{Check InOut Access Checks}
Είναι ο πατέρας των 2 παρακάτω κλάσεων, περιλαμβάνει τα κοινά τους checks ως προς το check in και check out των υπαλλήλων.
\subsubsection{Check In Access Checks}
Περιλαμβάνει έξτρα λειτουργίες που θα χρειάζεται για το check in των υπαλλήλων.
\subsubsection{Check Out Access Checks}
Περιλαμβάνει έξτρα λειτουργίες που θα χρειάζεται για το check out των υπαλλήλων, καθώς και έξτρα ελέγχους (προυπόθεση να έχει γίνει check in πριν, για παράδειγμα)
\subsection{Human Entity}
Η βασική έννοια-μοντελοποίηση της ανθρώπινης παρουσίας στο project. Περιλαμβάνει βασικές πληροφορίες καθώς και δικαιώματα πρόσβασης.
\subsubsection{Guest}
Σηματοδοτεί το μικρότερο επίπεδο πρόσβασης, χωρίς να υπάρχουν πολλά στοιχεία στη βάση δεδομένων. Χρησιμοποιείται για έκδοση προσωρινής κάρτας πρόσβασης.
\subsubsection{Employee}
Συμβολίζει κάποιον υπάλληλο που θα έχει ημιμόνιμη παρουσία σοτ σύστημα με τακτικά check in και out, κατά την καθημερινή του εργασία.
\subsubsection{Security Guy}
Το κλασσικό προσωπικό ασφαλέιας, από τον φρουρό που κάθεται στην πόρτα μέχρι και τους ανώτερους επικεφαλείς. Ο καθένας έχει ένα PIN assigned και μπορεί να κάνει τις περισσότερες λειτουργίες στο σύστημα όπως να καταχωρήσει περιστατικό, να βγάλει κάρτα επισκέπτη ή κάρτα εργαζομένου.

\subsubsection{Chief Security Guy}
Ελάχιστοι υπάλληλοι προσωπικού είναι σε τέτοιο επίπεδο δικαιωμάτων. 

\subsection{Employee Shift}
Συμβολίζει την βάρδια κάθε εργαζομένου (ώρες, ημέρες). Μπορεί να χρησιμοποιηθεί και για τον έλεγχο πρόσβασης προσωρινής κάρτας guest.

\subsection{Employee Schedule}
Περιλαμβάνει το πρόγραμμα βαρδιών όλων των υπαλλήλων της εταιρίας. Θεωρούμε ότι προυπάρχει τέτοιο σύστημα διαχείρησης πόρων στην εταιρία και απλά κάνουμε integrate με αυτό.

\subsection{Notification Subsystem}
Συμβολίζει το 3rd party σύστημα ελέγχου της περιμέτρου από κάμερες και αισθητήρες.

\subsection{Drone}
Συνδέει το λογισμικό μας μαζί με κάποιο άλλο έτοιμο drone kit για enterprise χρήση που αναφέραμε και στο πλάνο του έργου (π.χ. skydio.com). Επικοινωνεί με το κομμάτι καταχώρησης περιστατικού ασφαλείας καθώς και κάποιες έξτρα λειτουργίες-εντολές όπως να πάει σε ένα σημείο, να γυρίσει στη βάση ή να κάνει image recognition.
\subsection{Incident Classifier}
Νευρονικό δίκτυο το οποίο έχει εκπαιδευτεί κατάλληλα έτσι ώστε να μπορεί να αναγνωρίζει το είδος του περιστατικού μέσω φωτογραφίας.
\subsubsection{People Classifier}
Κατάλληλα εκπαιδευμένο νευρικό δίκτυο το οποίο έχει ως έξοδο την πιθανότητα το καταγεγραμμένο άτομο να έχει παραβατική συμπεριφορά, καθώς και την τοποθεσία του και λοιπά χαρακτηστικά.
\subsubsection{Animal Classifier}
Αντίστοιχα με το People Classifier μπορεί να αναγνωρίζει το είδος το ζώου.
\subsubsection{Fire Classifier}
Δέχεται φωτογραφίες ανά τακτά χρονικά διαστήματα και σύμφωνα με αυτές εντοπίζει εάν υπάρχει φωτιά, καπνού και την τοποθεσία της.
\subsubsection{Vehicle Classifier}
Δέχεται τις φωτογραφίες ως είσοδο και τις ταξινομεί ανάλογα με το είδος του αυτοκινήτου (π.χ. τετρακίνητο, φορτηγό κτλ.).

\subsection{Central Office Bridge}
Το κύριο σύστημα επικοινωνίας μας με το προ-υπάρχων σύστημα της εταιρίας της οποίας πουλάμε το προϊόν μας. Στέλνει τα δεδομένα του κάθε περιστατικού.

\subsection{Incident Submission}
Το σύστημα που ο κάθε security guy μπορεί να αναφέρει περιστατικά. Προαιρετικά, εάν υπάρχει κάποιος συναγερμός ή ειδοποίηση από το σύστημα ασφαλείας περιμέτρου, ζητάει από τον αρμόδιο εκείνη τη στιγμή να καταχωρίσει περιστατικό αυτόματα.
\subsection{Access Log}
Ένα μικρό σύστημα στο οποίο δικαιώματα πρόσβασης έχει το προσωπικό ασφαλείας. Από εδώ γίνεται προβολή των τελευταίων περιστατικών καθώς και υπάρχουν φίλτρα αναζήτησης.
\subsection{Security Incident}
Βασικές πληροφορίες περιστατικού ασφαλείας που καταχωρούνται.
\subsection{Access Log Data}
Έξτρα δεδομένα ή ψηφιακά έγγραφα που μπορούν να επισυναυθούν στο security incident entity.

\section{Domain Model Diagram}
Για δική σας ευκολία παρακαλώ να δείτε πέρα από τα screenshots το παρακάτω view-only link από draw.io:

\url{https://drive.google.com/file/d/1FmY-BqnuZeuAcTOVssY6idotpupvIt92/view?usp=sharing}

\noindent\makebox[\textwidth]{\includegraphics[width=\paperwidth]{domain.png}}
\newpage

\newpage

\section{Εργαλεία}
Χρησιμοποιήθηκαν:
\begin{itemize}
    \item \LaTeX/Overleaf.com - Συγγραφή του παρόντος τεχνικού κειμένου
    \item Photoshop - Φωτογραφία Σελίδας Τίτλου
    \item draw.io - Σχεδιασμός διαγραμμάτων
\end{itemize}


\end{document}
